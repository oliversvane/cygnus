\documentclass[10pt,twocolumn]{article}

% Packages
\usepackage[utf8]{inputenc}
\usepackage{amsmath, amssymb}
\usepackage{graphicx}
\usepackage{booktabs}
\usepackage{hyperref}
\usepackage{caption}
\usepackage{authblk}
\usepackage{abstract}

% Title and author
\title{Your Paper Title Here}
\author[1]{Oliver Svane Olsen}
\author[2]{Daniel Rohrer Hansen}
\affil[1]{Co-Engineering Aps\\ \texttt{olol@coec.dk}}
\affil[2]{Co-Engineering Aps\\ \texttt{daha@coec.dk}}

\date{}

\begin{document}

\twocolumn[
\maketitle
\begin{onecolabstract}
\noindent
\textbf{Abstract:}  
Write your abstract here. Summarize your work in about 150-250 words.
\end{onecolabstract}
\vspace{1cm}
]

\section{Introduction}
Introduce the problem, motivation, and background. Include references if needed.

\section{Related Work}
Summarize previous research relevant to your topic.

\section{Methodology}
Describe your methods, models, or algorithms in detail.

\section{Data}
For the diarization models in this work, three datasets were used. The CHiME-6 Challenge dataset focuses on distant multi-microphone conversational speech in natural home environments, recorded during dinner parties, and provides a challenging scenario for unsegmented multispeaker speech recognition with open-source baselines \cite{watanabe2020chime}. The AMI Meeting Corpus contains 100 hours of multi-modal meeting recordings from both scenario-driven design team meetings and naturally occurring meetings, annotated with orthographic transcriptions and rich phenomena, all released under a Creative Commons Attribution 4.0 license \cite{carletta2006ami}. Lastly, VoxConverse is an audio-visual diarization dataset extracted from YouTube videos, offering multispeaker clips with detailed speaker diarisation annotations, useful for real-world diarization tasks \cite{chung2020spot}.

\section{Experiments}
Explain experimental setup, datasets, evaluation metrics, and results.

\section{Discussion}
Interpret results, analyze strengths and limitations.

\section{Conclusion}
Summarize findings and suggest future work.

\section*{Acknowledgments}
(optional) Thank collaborators, funding sources, etc.

\bibliographystyle{plain}
\bibliography{references}

\end{document}
